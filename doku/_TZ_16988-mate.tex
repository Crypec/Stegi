\documentclass[a4paper, 12pt]{article}

\usepackage[utf8]{inputenc}
\usepackage{subfig}
\usepackage[ngerman]{babel}
\usepackage{amsmath,amsfonts,amssymb}
\usepackage{csquotes}
\usepackage{lmodern}
\usepackage[T1]{fontenc}

\usepackage{graphicx}
\graphicspath{{./images/}}

\usepackage[margin=3.5cm]{geometry}

\usepackage[
    backend=biber,
    style=numeric,
    sortlocale=de_DE,
    url=true, 
    eprint=false,
]{biblatex}
\addbibresource{References.bib}

\usepackage[]{hyperref}
\hypersetup{
    colorlinks=true,
    citecolor=black,
    urlcolor=blue,
    linkcolor = black
}

\title{Über komplexe Zahlen und ihre Bedeutung in der modernen Mathematik}
\author{Simon Kunz}
\date{\today}

\pagenumbering{gobble}

\begin{document}

\subsubsection{Die Polarform}
Abgesehen von der Normalform lässt sich eine komplexe Zahl $z$ aber auch durch ihren Abstand zum Ursprung $|z|$ und dem Winkel zur reellen Achse festlegen.
Wir erhalten somit eine sekundäre Darstellung einer komplexen Zahl, die Polarform.
Gegeben ist die Zahl $z = a +bi$ dann kann der Abstand vom Ursprung, also der Betrag der Zahl z, $|z|$ mithilfe des Satzes von Pythagoras bestimmet werden $|z| = \sqrt{a^2 +b^2}$.
Der Winkel, ausgehend von der reellen Achse, rotiert gegen den Uhrzeigersinn und wird allgemein mit $\phi = arg(z)$ bezeichnet.
Er ist gegeben durch $\arctan{ (\frac{Im(z)}{Re(z)})}$, für ihn gilt $Re(z) \neq 0$ und $0 \leq \phi < 2\pi$. \\
Es folgt also $a = |z| * \cos{\phi}$ und $b = |z| * sin{\phi}$.
Diesen Umstand können wir nutzen um diese Werte folgendermaßen $ z = |z|(\cos{\phi} + i * \sin{\phi})$, wieder in die Normalform einzusetzen.
Wir erhalten so die trigonometrische oder Polarform der Zahl z. \cite{book_mod}

\subsubsection{Die Exponentialform}
Die Herleitung der folgenden Formel würde den Rahmen dieses Projekts leider sprengen, allerdings kann durch sogennante Taylor-Reihen \cite{tay_ser} recht anschaulich bewießen werden, dass: \\
$e^{i \phi} = \sin{(\phi)} + i * \cos{(\phi)}$ \\
Wir können folglich die oben beschriebene Polarform in die Exponentialform umwandeln, dies erleichtert vor allem das Rechnen mit komplexen Zahlen, z.B. für die Berechnung von Logarithmen ungemein leichter.

\subsection{Rechenoperationen}
\subsubsection{Addition}
Die Addition zweier reeller Zahlen funktioniert genau wie die Addition zweier Vektoren. Es werden jeweils Realteil $Re(a) + Im(b)$ der beiden Zahlen addiert. \cite{book_alt}
\[ (a_0 + b_0 i) + (a_1 + b_0 i) = a_0 + a_1 + b_0 + b_1 i \]

\begin{figure}[h!]%
    \centering

\end{document}
