\documentclass[a4paper, 12pt]{article}

\usepackage[utf8]{inputenc}
\usepackage{subfig}
\usepackage[ngerman]{babel}
\usepackage{amsmath,amsfonts,amssymb}
\usepackage{csquotes}
\usepackage{lmodern}
\usepackage[T1]{fontenc}

\usepackage{graphicx}
\graphicspath{{./images/}}

\usepackage[margin=3.5cm]{geometry}

\usepackage[
    backend=biber,
    style=numeric,
    sortlocale=de_DE,
    url=true, 
    eprint=false,
]{biblatex}
\addbibresource{References.bib}

\usepackage[]{hyperref}
\hypersetup{
    colorlinks=true,
    citecolor=black,
    urlcolor=blue,
    linkcolor = black
}

\title{Über komplexe Zahlen und ihre Bedeutung in der modernen Mathematik}
\author{Simon Kunz}
\date{\today}

\pagenumbering{gobble}

\begin{document}


\begin{titlepage}
   \begin{center}
       \vspace*{2cm}
 
       \textbf{Die Geschichte der verloren Nullstellen}
 
       \vspace{0.5cm}
       Über komplexe Zahlen und ihre Bedeutung in der modernen Mathematik
 
       \vspace{1.5cm}
 
       \textbf{Simon Kunz}
 
       \vspace{5cm}
         
       Ausarbeitung für meine GFS im Fach\\
       Mathematik
 
       \vspace{0.8cm}
 
       \vspace{0.8cm}

       Technisches Gymnasium Waldshut\\
       Deutschland\\
       28.06.2019

       simon.kunz@schueler.gs-wt.de
   \end{center}
\end{titlepage}



\clearpage
\tableofcontents
\clearpage

\pagenumbering{arabic}

\section{Vorwort}
Eine vollständige Version dieser Ausarbeitung können sie online unter folgender Adresse finden: https://github.com/Crypec/Komplexe-Zahlen

\section{Einführung}
Die Rolle die Computer in unserem Leben spielen nimmt von Jahr zu Jahr
zu. Mit dem Aufstieg des Internets, microchips und computern und immer
mehr und mehr Haushaltsgegenstaenden. Dabei sind Computer und
insbesondere die Programme welche auf ihnen Laufen fuer schon laengst
nicht mehr aus unsrem Leben wegzudenken.
Immer weiter integrien wir Computer und damit auch Software in die
intimsten Bereiche unseres Lebens. Aber wie funktioniert das
eigentlich? Wie wird ein Computer eigentlich programmiert?
Koennen Computer denn schon unsere Sprache sprechen um unseren
Befehlen folgen zu koennen?

Die Antwort auf die letzte Frage ist ein klares Jein. Computer koennen
unsere Sprache sprechen, allerdings weder Deutsch noch Englisch...
Um einem Computer Befehle erteilen zu koennen bedarf es einer
speziellen Sprache, einer sogennanten Programmiersprache.
Aber warum eigentlich? Warum koennen wir dem Computer nicht einfach
auf Deutsch sagen was wir von ihm wollen?
Die Antwort auf diese Frage wollen wir anhand folgendem Beispiel erlaeutern.

Stellen Sie sich vor sie wollten ein selbstfahrendes Auto bauen. Ein zum aktuellen Zeitpunkt zumutbares Projekt. Sie fangen nun an einige Regel zu fuer den Computer in ihrem Auto zu definieren.
Im Zuge dem Prozess dieser Regeldefinitionen koennten Sie auf folgende
Regel stossen:

Das Auto soll Fussgaenger grundsaetzlich umfahren.

\subsection{Zahlenmengen}

\subsubsection{Natürliche Zahlen}
Wenn wir nun Zahlen aus dem Raum natürlicher Zahlen $\in \mathbb{N}_0$ betrachten stellen wir fest:
Die Punkte zweier aufeinanderfolgenden Zahlen besitzen immer den Gleichen Abstand und sind weiterhin ein Vielfaches des Vektors $\vec{1}$. Gleichzeitig ist $0$ die kleinste Zahl die wir somit darstellen können und zudem hat sie als einzige Zahl auf keinen Vorgänger.
Allerdings sind wir mit den Rechenoperationen die wir anwenden können beschränkt. Nur Multiplikation und Addition können uneingeschränkt genutzt werden. \cite{mengen}

\subsubsection{Ganze Zahlen}
Wir können zumindest das Problem der Subtraktion mithilfe der ganzen Zahlen lösen, indem wir davon ausgehen dass es für alle $a + b = 0$, wobei $a > 0$ eine Lösung gibt.
Somit haben wir nun die ganzen Zahlen definiert, sie besitzen allerdings fast die gleichen Eigenschaften wie die natürlichen Zahlen.
Negativen Zahlen ordnen wir einen nach links gerichteten Vektoren zu. Wir können nun also zwei Zahlen haben welche den gleichen \glqq Wert\grqq \ oder Abstand von $(0,0)$ z.B. $|-a| = |a|$ besitzen, aber nicht der  Zahl entspreichen. $a\ und -a$

\subsubsection{Reelle Zahlen}
Die reellen Zahlen schließen die jetzt noch vorhandene Lücken in der Division. Wir können nun alle Zahlen, welche wir aus der Schule kennen als einem einfachen 1 Dimensionalen Vektor darstellen.

\section{Imaginäre Zahlen}
Allerdings reichen selbst die reellen Zahlen nicht aus um alle Gleichungen uneingeschränkt lösen zu können. 
Dies kann am besten an folgendem Beispiel gezeigt werdem.
Gegeben ist die folgende Gleichung: $x^2 - 1 = 0$ sie zu lösen sollte kein Problem sein.
Wir schieben hierzu einfach die $- 1$ auf die andere Seite des Gleichheitszeichen,  die Gleichung sieht nun folgendermaßen aus: $x^2 = 1$.
Es bleibt nur noch die Wurzel von 1 zu ziehen.
Die Lösung der Gleichung ist nun also: $x = 1$
Anders sieht es allerdings wenn sie folgende Gleichung lösen sollen: $x^2 + 1 = 0$.
Wir gehen nun also nach dem gleichen Prinzip wie oben vor, nehmen die 1 auf die andere Seite $x^2 = -1$, jetzt nur noch die Wurzel ziehen $x = \sqrt{-1}$.
Wir stehen nun vor einem Problem, vor dem uns die Mathelehrer der letzen Jahre immer genaustens gewarnt haben.
Wir können keine Wurzel einer negativen Zahl ziehen, denn $a * a = a^2$ und $-a * -a = a^2$. Es scheint keine Möglichkeit zu geben eine Zahl zu erhalten welche mit sich selbst multiplziert wiederum eine negative Zahl ergibt.
Das heißt, müssen eine neue Zahl definieren welche diese Eigenschaft erfüllt, so dass sie mit sich selbst multiplziert eine negative Zahl ergibt. Dies gelang erstmals den Mathematikern Caspar Wessel und Rowan William Hamilton \cite{i_def}. Später führte Leonard Euler die heute noch gebräuliche Definition $i^2 = -1$ ein. Geboren sind die imaginären Zahlen. Sie erlauben uns unter anderem die Wurzel negativer Zahlen zu ziehen: 

\begin{equation} \label{neg_sqrt}
  \begin{split}
   & \sqrt{-a} = \sqrt{-1} * \sqrt{a} = \sqrt{a}i \\
   & \sqrt{-9} = \sqrt{-1} * \sqrt{9} = 3i
  \end{split}
\end{equation}

\section{Komplexe Zahlen}

\subsection{Einführung in die Welt der komplexen Zahlen}

\subsubsection{Die Darstellung in der Gaußschen Zahlenebene}
Wir verbinden nun die imaginären Zahlen mit den reellen Zahlen um die komplexen Zahlen zu erhalten. Diese nehemen typischerweiße die Form $ z \in \mathbb{C}= z = a + bi$ an.
Um sie graphisch darzustellen reicht die einfache Zahlengerade von Anfangs nicht mehr aus. Wir müssen sie um eine weitere, (Imaginäre) Achse erweitern.
Nun können wir die komplexen Zahlen ähnlich einer Kordinate als Punkt in eben diesem Kordinatensystem darstellen (Dieses wird auch Gaussche Zahlenebene genannt). Der Realteil der Zahl entspricht dabei der Verschiebung auf der reellen Achse, der Imaginärteil der Zahl entspricht dann der Verschiebung auf der imaginäre Achse. \cite{book_mod}
Der Zusammenschluss der Beiden Zahlen entsprich also stark vereinfacht einem Punkt auf einer 2 Dimensionalen Plane \cite{book_alt}.


\subsubsection{Die Polarform}
Abgesehen von der Normalform lässt sich eine komplexe Zahl $z$ aber auch durch ihren Abstand zum Ursprung $|z|$ und dem Winkel zur reellen Achse festlegen.
Wir erhalten somit eine sekundäre Darstellung einer komplexen Zahl, die Polarform.
Gegeben ist die Zahl $z = a +bi$ dann kann der Abstand vom Ursprung, also der Betrag der Zahl z, $|z|$ mithilfe des Satzes von Pythagoras bestimmet werden $|z| = \sqrt{a^2 +b^2}$.
Der Winkel, ausgehend von der reellen Achse, rotiert gegen den Uhrzeigersinn und wird allgemein mit $\phi = arg(z)$ bezeichnet.
Er ist gegeben durch $\arctan{ (\frac{Im(z)}{Re(z)})}$, für ihn gilt $Re(z) \neq 0$ und $0 \leq \phi < 2\pi$. \\
Es folgt also $a = |z| * \cos{\phi}$ und $b = |z| * sin{\phi}$.
Diesen Umstand können wir nutzen um diese Werte folgendermaßen $ z = |z|(\cos{\phi} + i * \sin{\phi})$, wieder in die Normalform einzusetzen.
Wir erhalten so die trigonometrische oder Polarform der Zahl z. \cite{book_mod}

\subsubsection{Die Exponentialform}
Die Herleitung der folgenden Formel würde den Rahmen dieses Projekts leider sprengen, allerdings kann durch sogennante Taylor-Reihen \cite{tay_ser} recht anschaulich bewießen werden, dass: \\
$e^{i \phi} = \sin{(\phi)} + i * \cos{(\phi)}$ \\
Wir können folglich die oben beschriebene Polarform in die Exponentialform umwandeln, dies erleichtert vor allem das Rechnen mit komplexen Zahlen, z.B. für die Berechnung von Logarithmen ungemein leichter.

\subsection{Rechenoperationen}
\subsubsection{Addition}
Die Addition zweier reeller Zahlen funktioniert genau wie die Addition zweier Vektoren. Es werden jeweils Realteil $Re(a) + Im(b)$ der beiden Zahlen addiert. \cite{book_alt}
\[ (a_0 + b_0 i) + (a_1 + b_0 i) = a_0 + a_1 + b_0 + b_1 i \]

Ein Beispiel wäre:
\begin{equation} \label{add_ex1}
  \begin{split}
        K & = (4 + 2i) + (-3  + 5i) = 1 + 7i
  \end{split}
\end{equation}


\subsubsection{Subtraktion}
Die Subraktion zweier komplexer Zahlen in ihrer kartesischen Darstellung verhällt sich exakt wie die Addition, es werden nur Realteil und Imaginärteil voneinander abgezogen.
\[ (a_0 + b_0 i) + (a_1 + b_0 i) = a_0 + a_1 + b_0 + b_1 i \]




\subsubsection{Multiplikation}
Um komplexe Zahlen miteinandere zu multiplizieren gibt es 2 Möglichkeiten. \\
Die Multiplikation in der Normalform:
Hier werden die beiden Zahlen, ähnlich dem Rechnen mit reellen Zahlen ausmultipliziert.
\begin{equation} \label{mul_ex1}
  \begin{split}
        K & = (4 - 2i) * (-3  + 5i) \\
        & = 4 * (-3) + 4 * 5i + (-2i) * (-3) + (-2i) * (-2i) * 5i \\
        & = -12 + 20i + 6i - 10i^2 \\
        & = -2 + 26i
  \end{split}
\end{equation}

Oder die Multiplikation in der Polarform. Hier können die Potenzrechengesetzte angewendet werden.
\begin{equation} \label{mul_ex2}
  \begin{split}
    K & = 4 * e^{\pi i} * 3 * e^{\frac{\pi}{2}i} \\
    & = (4 * 3) * e^{\pi i + {\frac{\pi}{2}i}}\\
    & = 7 * e^{\frac{3}{2} \pi i}\\
  \end{split}
\end{equation}

\subsubsection{Division}
Ähnlich der Multiplikation kann die Division mit komplexen Zahlen je nach vorligender Form ausgeführt werden.
In der Normalform wird der Bruch mit der komplex konjugierten Version des Nenners erweitert. Das ist valide, weil hier eigentlich nur eine konstante 1 steht, welche denn Wert der Gleichug also nicht verändert, allerdings sicherstellt, dass sich eine reelle Zahl im Nenner befindet. 
Um eine komplexe Zahl zu konjugieren wird nur das Vorzeichen des Imaginärteils umgedreht. 
$z = a +bi$ wird zu  $\overline{z} = a - bi$.
Anschließend wird mit der komplex konjugierten Zahl erweitert und multiplziert. 
\begin{equation} \label{div_ex1}
  \begin{split}
    K & = \frac{3 + 2i}{1 + 4i} \\
    & = \frac{3 + 2i}{1 + 4i} * \frac{1 -4i}{1 - 4i} \\
    & = \frac{11}{17} - \frac{10}{17}i
 \end{split}
\end{equation}
Die Division in der Polarform entspricht dabei exakt der Multiplikation in Exponentialform.
\begin{equation} \label{div_ex2}
  \begin{split}
    K & = r_0 * e^{i \phi_0}/r_1 * e^{i \phi_1} \\
    & = \frac{r_0}{r_1} * e^{i (\phi_0 - \phi_1))} \\
 \end{split}
\end{equation}

\section{Komplexe Funktionen}
\subsection{Der Komplexe Logarithmus}
Um den komplexen Logarithmus einer Zahl $z \in \mathbb{C}$ zu berchnen, können wir wie oben schon erwähnt die Exponentialform einer komplexen Zahl verwenden. Allerdings müssen wir eine Sache beachten.
Der Vektor einer komplexen Zahl in der Exponentialform kann für verschiedene Winkel von $\phi$ identisch sein. Immer wenn $\phi$ eine volle Umdrehung von $2 \pi$ macht, $\phi  +2\pi * k$ mit $k \in \mathbb{Z}$ befinden wir uns wieder an dem Punkt an dem wir gestarted sind.
Wir können die Exponentialform also folgendermaßen umschreiben:
\[z = r * e^{i\phi +  2\pi * k}\]
Nun können wir uns die normalen Logarithmusgesetze zur Hilfe nehmen. Es gilt:
\[ln(z) = ln(r * e^{i\phi +  2\pi * k}) = ln(r) + ln(e^{i\phi +  2\pi * k})\]
Der Logarithmus ist die Umkehrfunktion der Potenz, es folgt also $ln(e^x) = x$.
\[ln(z) = ln(r * e^{i\phi +  2\pi * k}) = ln(r) + ln(e^{i\phi +  2\pi * k}) = ln(r) + \phi + 2\pi * k\]
Hier wird deutlich, der Logarithmus liefert kein eindeutiges Ergebnis mehr, für den logarithmus $ln(z)$ existieren unendlich viele Lösungen, welche sich alle nur um den Faktor $2 \pi * k$ unterscheiden. \cite{komp_log}


\section{Eigenständigkeitserklärung}
Hiermit erkläre ich, dass ich die vorliegende Hausarbeit selbständig verfasst und keine anderen als die angegebenen Hilfsmittel benutzt habe.
Die Stellen der Hausarbeit, die anderen Quellen im Wortlaut oder dem Sinn nach entnommen wurden, sind durch Angaben der Herkunft kenntlich gemacht. Dies gilt auch für Zeichnungen, Skizzen, bildliche Darstellungen sowie für Quellen aus dem Internet.
\\\\\\\\

Simon Kunz (26.06.2019)

\newpage
\printbibliography

\end{document}
